\documentclass[12pt]{article}

\usepackage{verbatim}
\usepackage{listings}
\usepackage{color}


%%
\title{An educationnal HBRIDGE board}

\author{Fabien Le Mentec\\
\small \texttt{texane@gmail.com}
}

\date{}

\begin{document}
\maketitle

\begin{abstract}
This document describes an educationnal HBRIDGE board whose prime goal is to learn more about alectronics.
\end{abstract}


%% Introduction
\section{Introduction}

\subsection{Purpose}
\paragraph{} I wanted a small project to learn more about electronics basics. Since I regularly need to
drive a DC motors, I chose to work on a home made HBRIDGE board described in this document. Note that
the circuit uses off the shelf parts and can be more complicated than needed and not efficient. A real
application should use packaged HBRIDGE circuits.


%% Features
\section{Features}
\paragraph{} The board features:
\begin{itemize}
  \item safe H signalling interface avoid conflicting states
    \begin{itemize}
      \item PWM,
      \item FORWARD,
      \item BRAKE.
    \end{itemize}
  \item controlling software
    \begin{itemize}
      \item coding done by XXX
    \end{itemize}
  \item power stage driving up to XXX motors.
\end{itemize}


%% Theory of operation
\section{HBRIDGE theory of operation}
\paragraph{} An HBRIDGE is a circuit allowing to drive DC motors in both direction. The name comes from the
circuit routing which looks like the 'H' alphabet letter.


%% Safe H signalling interface
\section{Safe H signalling interface}

\paragraph{} Since I have a software background it is easier for me to think in terms of C programming control
structures. I thus expressed the H state machine in a C code which I translated into logical statement. I used
this statement set to design the H state circuit using electronic gates.

\begin{tiny}
\lstset{commentstyle=\color{blue}}
\lstset{language=C}
\begin{lstlisting}[frame=tb]
if (BRAKE == 0)
{
  if (FWD == 1)
  {
    Q01 = 0;
    Q10 = 0;
    Q11 = 1;

    Q00 = PWM;
  }
  else /* REVERSE */
  {
    Q00 = 0;
    Q01 = 1;
    Q11 = 0;

    Q10 = PWM;
  }
}
else /* (BRAKE == 1) */
{
  Q00 = 1;
  Q01 = 1;
  Q10 = 1;
  Q11 = 1;
}
\end{lstlisting}
\end{tiny}

The set of logical statments reduces to:
\begin{tiny}
\lstset{commentstyle=\color{blue}}
\lstset{language=C}
\begin{lstlisting}[frame=tb]
Q00 = (FWD & PWM)    | BRAKE;
Q01 = (!FWD)         | BRAKE;
Q10 = ((!FWD) & PWM) | BRAKE;
Q11 = FWD            | BRAKE;
\end{lstlisting}
\end{tiny}

The following circuit implements it:


%% Bill of materials
\section{Bill of materials}
TODO


%% Status
\section{Status}
\begin{itemize}
  \item PWM / FORWARD / BRAKE signaling interface: PROTOTYPED
  \item control software: TODO
  \item power stage: TODO
  \item documentation: STARTED
\end{itemize}


%% Conclusion
\section{Conclusion}
TODO


%% References
\section{Further readings}

\subsection{HBRIDGE projects}
\begin{itemize}
  \item http://embedded-lab.com/blog/?p=1159
  \item http://www.mcmanis.com/chuck/robotics/tutorial/h-bridge
  \item http://www.modularcircuits.com/h-bridge\_secrets1.htm
  \item http://www.solarbotics.net/library/circuits/driver\_4varHbridge.html
  \item http://www.solarbotics.net/library/circuits/driver\_buf\_h.html
  \item http://www.robotroom.com/HBridge.html
  \item http://www.robotroom.com/BipolarHBridge.html
\end{itemize}

\subsection{Controlling software}
\begin{itemize}
  \item http://www.seattlerobotics.org/encoder/200001/simplemotor.htm
\end{itemize}


%% TODO
\end{document}
